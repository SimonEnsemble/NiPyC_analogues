\documentclass[11pt]{article}

\usepackage{amsmath}
\usepackage{siunitx}
\usepackage{natbib}

\newcommand{\Dbar}{{\mathchar'26\mkern-11mu\mathrm{D}}}


\author{Sondre K. Schnell}
\title{\texttt{tcf}-Time Correlation Functions}

\begin{document}
\maketitle

\begin{abstract}

\texttt{tcf} is a C-program to calculate time correlation functions.
To study self- and collective diffusion using computer simulations, we
need to track the movement of particles, or groups of particles over
time. This program allows us to post-process a trajectory from Molecular
Dynamics simulations. The correlation functions are calculated based on
the order-N algorithm originally found in Frenkel and
Smit~\cite{fre021}, modified by Dubbeldam \textit{et al.}~\cite{dub091}.

\end{abstract}

\section{Diffusion}
 

There are several different types of diffusion coefficients. In this
case however, we will distinguish between two diffent types: (1) the
self-diffusion coefficient, $D$, and (2) the collective diffusion
coefficient, $\Dbar$. Both of these phenomena are important to
understand, and to describe in a good way. 


We are talking about two different types of diffusion: Self- and
collective diffusion. The self-diffusion is the easiest type to
understand. We can calculate this in two different ways from Molecular
Simulations: (1) from the mean-squared displacement, and (2) from the
velocity autocorrelation function. 

The self-diffusion coefficient is best described from the change in
concentration due to a gradient in concentration:
\begin{equation}
\frac{\partial c}{\partial t} = D \nabla ^2 c
\end{equation}
where $c$ is the concentration, $t$ is the time, and $D$ is the
self-diffusion coefficient. From Einstein we know that the
self-diffusion coefficient can be related to the mean-squared
displacement: 
\begin{equation}
\frac{\partial \langle r^2 \rangle}{\partial t} = 2dD
\end{equation}
where $\langle r^2 \rangle$ is the mean-squared displacement, and $d$ is the
dimensionality. For long times, the self-diffusion coefficient is
independent of time, thus
\begin{equation}
\langle r^2 \rangle = \lim_{t\rightarrow \infty} 2d D t. 
\end{equation}

The diffusion coefficient should only be calculated for systems using
the mean squared displacement when it is larger than
\SI{1e-12}{\kelvin}
this, we have to use a different approach to discribing the diffusion. 




\bibliographystyle{unsrt}
\bibliography{/Users/sondresc/Dropbox/references/refdb2}

\end{document}
